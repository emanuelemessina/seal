\documentclass[10pt,twocolumn,letterpaper]{article}

\usepackage{cvpr}
\usepackage{times}
\usepackage{epsfig}
\usepackage{graphicx}
\usepackage{amsmath}
\usepackage{amssymb}

% Include other packages here, before hyperref.

% If you comment hyperref and then uncomment it, you should delete
% egpaper.aux before re-running latex.  (Or just hit 'q' on the first latex
% run, let it finish, and you should be clear).
\usepackage[breaklinks=true,bookmarks=false]{hyperref}

\cvprfinalcopy % *** Uncomment this line for the final submission

\def\cvprPaperID{****} % *** Enter the CVPR Paper ID here
\def\httilde{\mbox{\tt\raisebox{-.5ex}{\symbol{126}}}}

\setcounter{page}{1}
\begin{document}

%%%%%%%%% TITLE
\title{SEAL: Structured Encoder for Ancient Logograms}
\date{2025}

\author{Emanuele Messina\\
{\tt\small giuseppeemanuele.messina@studenti.polito.it}
\and
Salvatore Cimmino\\
{\tt\small salvatore.cimmino@studenti.polito.it}
}

\maketitle

%%%%%%%%% ABSTRACT
\begin{abstract}
  Most OCR softwares do not support chinese seal script characters, which hold significant cultural and historical value. In this paper, we present our contributions towards developing a custom detector capable of recognizing thousands of these characters, along with a comprehensive dataset. Our model leverages the intrinsic hierarchical nature of chinese characters to reduce the latent space dimension compared to a standard classifier. This allows for a more compact model with feasible training and inference times while achieving reasonable accuracy on 5684 classes -- while a non hierarchical version of our model and YOLO11x could not converge at all.
\end{abstract}

%%%%%%%%% BODY TEXT

\section{Introduction}
\label{sec:introduction}

TODO:
 - talk about problem,  what is it, why we do it, a bit of history/ how it works, why we want to use hierarchical classification
 

\section{Data}
\label{sec:data}

TODO:

- talk about dataset, how we constructed it, what it contains, the label structure, show distributions

https://www.kaggle.com/datasets/emanuelemessina/seal-5684/data
\section{Methods}
\label{sec:methods}

TODO:


- talk about the model, we used fasterrcnn cuz its easy to customize, and also we conjecture 2 stage is better than single shot (see yolo results)
- show how fasterrcnn works and why it doesn't work for use, how we solved the problem for 5k classes
- explain custom model, loss, class weights

\section{Experiments}
\label{sec:experiments}

TODO:

- tell about linear model, distance from features,  725 - 2061 - 5684, regressor vs classifier, adam fails, lr, scheduler, anchors, epochs



For our initial tests of the mode, we used reduced versions of our dataset based on the first 1k and then 3k most frequent characters according to a classical chinese corpus frequency list, before moving to the complete set.


\section{Results}
\label{sec:results}

TODO:


- results, losses, hc vs no hc, yolo

{\small
  \bibliographystyle{ieee_fullname}
  \bibliography{egbib}
}

\end{document}
